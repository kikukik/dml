\begin{task}[credit=16]{Entscheidungsbäume - ID3 Algorithmus}
%\exercise{Entscheidungsbäume - ID3 Algorithmus}
%The table below shows making decision of playing baseball or not, based on four weather attributes.
Die folgende Tabelle zeigt die Entscheidung, ob Baseball gespielt wird, basierend auf vier Wetterattributen.

\begin{table}[h]
\centering
\caption{Trainingsdatensatz, ob Baseball gespielt wird basierend auf der Wetterlage.}
\label{tab:data_baseball}
\begin{tabular}{l|c|c|c|c|c}
\toprule
\textbf{Tag} & \textbf{Ausblick (A)} & \textbf{Temperatur (T)}  & \textbf{Luftfeuchtigkeit (L)} & \textbf{Wind (W)}     & \textbf{Spielt Baseball (B)} \\
\midrule
T1  & Sonnig    & Warm        & Hoch             & Schwach  & Nein            \\
T2  & Sonnig    & Warm        & Hoch             & Stark    & Nein            \\
T3  & Bewölkung & Warm        & Hoch             & Schwach  & Ja              \\
T4  & Regen     & Mild        & Hoch             & Schwach  & Ja              \\
T5  & Regen     & Kühl        & Normal           & Schwach  & Ja              \\
T6  & Regen     & Kühl        & Normal           & Stark    & Nein            \\
T7  & Bewölkung & Kühl        & Normal           & Stark    & Ja              \\
T8  & Sonnig    & Mild        & Hoch             & Schwach  & Nein            \\
T9  & Sonnig    & Kühl        & Normal           & Schwach  & Ja              \\
T10 & Regen     & Mild        & Normal           & Schwach  & Ja              \\
T11 & Sonnig    & Mild        & Normal           & Stark    & Ja              \\
T12 & Bewölkung & Mild        & Hoch             & Stark    & Ja              \\
T13 & Bewölkung & Warm        & Normal           & Schwach  & Ja              \\
T14 & Regen     & Mild        & Hoch             & Stark    & Nein            \\
\bottomrule
\end{tabular}
\end{table}

\begin{table}[h!]
\centering
\caption{Vorhersage-Datensatz, ob Baseball gespielt wird.}
\label{tab:data_baseball_predict}
\begin{tabular}{l|c|c|c|c|c}
\toprule
\textbf{Tag} & \textbf{Ausblick (A)} & \textbf{Temperatur (T)}  & \textbf{Luftfeuchtigkeit (L)} & \textbf{Wind (W)}     & \textbf{Spielt Baseball (B)} \\
\midrule
T15 & Sonnig    & Mild        & Hoch     & Schwach & ?       \\
T16 & Bewölkung & Mild        & Normal   & Schwach & ?       \\
T17 & Regen     & Kühl        & Normal   & Stark   & ?       \\
\bottomrule
\end{tabular}
\end{table}

%The target is to answer the question ``Should we play baseball?''
Die Aufgabe ist es folgende Frage zu beantworten: \textit{Unter welchen Bedingungen wir Baseball gespielt?}

%\begin{questions}
%\begin{question}{ID3 Algorithmus}{10}
\begin{subtask}[points=10,title=ID3 Algorithmus]
\label{q:id3_alg}
Erstellen Sie den Entscheidungsbaum mittels des ID3 Algorithmus.
Berechnen Sie dabei die \textbf{Entropie} und den \textbf{Informationsgewinn} (engl. \textit{gain}) der Attribut-Selektion für jeden Schritt.

\begin{solution}
Wir filtern im ersten Schritt nach Attributen und Schätzen $p_+$ und $p_-$ für jede Ausprägung des Attributs durch Zählen. Es ergibt sich
\begin{table}[h!]
	\centering
	\caption{Attribut: Ausblick}
	\begin{tabular}{l|c|c|c|l}
		\toprule
		\textbf{Ausprägung} & \textbf{Anzahl} & \textbf{davon +}  & \textbf{davon -} &\textbf{Entropy} \\
		\midrule
		Sonnig    & 5 &2&3&0.971   \\
		Bewölkt & 4&4&0&0  \\
		Regen    & 5&3&2&0.971    \\
		\bottomrule
	\end{tabular}
\end{table}
\begin{table}[h!]
	\centering
	\caption{Attribut: Temperatur}
	\begin{tabular}{l|c|c|c|c}
		\toprule
		\textbf{Ausprägung} & \textbf{Anzahl} & \textbf{davon +}  & \textbf{davon -} &\textbf{Entropy} \\
		\midrule
		Warm  & 4 &2&2&1      \\
		Mild & 6&4&2&0.918   \\
		Kühl & 4&3&1&0.811    \\
		\bottomrule
	\end{tabular}
\end{table}
\begin{table}[h!]
	\centering
	\caption{Attribut: Luftfeuchtigkeit}
	\begin{tabular}{l|c|c|c|c}
		\toprule
		\textbf{Ausprägung} & \textbf{Anzahl} & \textbf{davon +}  & \textbf{davon -} &\textbf{Entropy} \\
		\midrule
		Hoch  & 7 &3&4&0.985      \\
		Normal & 7&6&1&0.591   \\
		\bottomrule
	\end{tabular}
\end{table}
\begin{table}[h!]
	\centering
	\caption{Attribut: Wind}
	\begin{tabular}{l|c|c|c|c}
		\toprule
		\textbf{Ausprägung} & \textbf{Anzahl} & \textbf{davon +}  & \textbf{davon -} &\textbf{Entropy} \\
		\midrule
		Stark  & 6 &3&3&1      \\
		Schwach &8&6&2&0.811   \\
		\bottomrule
	\end{tabular}
\end{table}
\newline
Pro Ausprägung setzen wir nun
\begin{align*}
p_+=\frac{\text{davon +}}{\text{Anzahl}}\;\;\; \text{und}\;\;\; p_-=\frac{\text{davon -}}{\text{Anzahl}}
\end{align*}
und berechnen die Entropien mittels $I(p_+,p_-)=(-p_+\cdot\log p_+)+(-p_-\cdot\log p_-)$. Die Ergebnisse haben wir an die Tabelle angefügt. Letztlich finden wir Für jedes Attribut Att. den Informationsgehalt mittels \begin{align*}
\text{Information}(\text{Att.})=\sum_{\text{Ausprägung A von Att.}}\frac{\text{Anzahl(A)}}{14}\cdot\text{Entropy(A)}
\end{align*}
Es ergibt sich
\end{solution}

\end{subtask}

\begin{subtask}[points=3,title=Visualisierung]
Erstellen Sie eine Visualisierung (Plot oder eingefügte Zeichnung) des Entscheidungsbaumes aus Aufgabenteil~\ref{q:id3_alg}.

\begin{solution}
% Geben sie hier ihre Antwort an.
\end{solution}

\end{subtask}

\begin{subtask}[points=3,title=Vorhersage]
Geben Sie anhand ihres Entscheidungsbaumes eine Vorhersage für die Tage 15 bis 17 aus Tabelle~\ref{tab:data_baseball_predict}, ob Baseball gespielt wird.

\begin{solution}
% Geben sie hier ihre Antwort an.
\end{solution}

\end{subtask}
\end{task}